% Этот шаблон документа разработан в 2014 году
% Данилом Фёдоровых (danil@fedorovykh.ru) 
% для использования в курсе 
% <<Документы и презентации в \LaTeX>>, записанном НИУ ВШЭ
% для Coursera.org: http://coursera.org/course/latex .
% Исходная версия шаблона --- 
% https://www.writelatex.com/coursera/latex/3.1
\documentclass[a4paper,14pt]{article}
\usepackage{extsizes}

%%% Работа с русским языком
\usepackage{cmap}					% поиск в PDF
\usepackage{mathtext} 				% русские буквы в формулах
\usepackage[T2A]{fontenc}			% кодировка
\usepackage[utf8]{inputenc}			% кодировка исходного текста
\usepackage[english,russian]{babel}	% локализация и переносы

%%% Дополнительная работа с математикой
\usepackage{amsmath,amsfonts,amssymb,amsthm,mathtools} % AMS
\usepackage{icomma} % "Умная" запятая: $0,2$ --- число, $0, 2$ --- перечисление

%% Номера формул
%\mathtoolsset{showonlyrefs=true} % Показывать номера только у тех формул, на которые есть \eqref{} в тексте.
%\usepackage{leqno} % Нумерация формул слева

%% Свои команды
\DeclareMathOperator{\sgn}{\mathop{sgn}}

%% Перенос знаков в формулах (по Львовскому)
\newcommand*{\hm}[1]{#1\nobreak\discretionary{}
{\hbox{$\mathsurround=0pt #1$}}{}}

%%% Работа с картинками
\usepackage{graphicx}  % Для вставки рисунков
\graphicspath{{images/}{images2/}}  % папки с картинками
\setlength\fboxsep{3pt} % Отступ рамки \fbox{} от рисунка
\setlength\fboxrule{1pt} % Толщина линий рамки \fbox{}
\usepackage{wrapfig} % Обтекание рисунков текстом

%%% Работа с таблицами
\usepackage{array,tabularx,tabulary,booktabs} % Дополнительная работа с таблицами
\usepackage{longtable}  % Длинные таблицы
\usepackage{multirow} % Слияние строк в таблице

%%% Теоремы
\theoremstyle{plain} % Это стиль по умолчанию, его можно не переопределять.
\newtheorem{theorem}{Теорема}[section]
\newtheorem{proposition}[theorem]{Утверждение}
 
\theoremstyle{definition} % "Определение"
\newtheorem{corollary}{Следствие}[theorem]
%\newtheorem{problem}{Задача}[section]
 
\theoremstyle{remark} % "Примечание"
\newtheorem*{nonum}{Решение}

%%% Программирование
\usepackage{etoolbox} % логические операторы

%%% Заголовок
\author{\LaTeX{} в Вышке}
\title{3.1 Счетчики и макрокоманды}
\date{\today}

\begin{document} % конец преамбулы, начало документа

\maketitle

\begin{itemize}
	\item Первый пункт
	\item Второй пункт
	\begin{itemize}
		\item Первый подпункт
		\begin{enumerate}
			\item Первый подподпункт
			\item Второй подподпункт
		\end{enumerate}
		\item Второй подпункт
		\item Третий подпункт
	\end{itemize}
	\item Третий пункт
\end{itemize}

\newcommand{\ti}[1]{\textit{#1}} 

\ti{arg1}

\renewcommand{\thesection}{\Alph{section}}
\renewcommand{\thesubsection}
{\thesection\arabic{subsection}}

\section{Введение}
\section{Основаная часть}
\subsection{Начало}
\subsection{Середина}
\subsection{Окончание}
\section{Заключение}

\addcontentsline{toc}{subsection}{Дополнительный подраздел} 
\addcontentsline{toc}{section}{Дополнительный подраздел} 
\tableofcontents

\newcommand{\problem}[1]{\par\bigskip\noindent
	\textbf{Решение задачи #1.}\enskip\ignorespaces}

\problem{1} Здесь написано решение задачи 1.


\end{document} % конец документа

